 What is gAttrib? 

  gattrib is the attribute editor for gschem, the schematic capture program/tool which is part of gEDA. gattrib's sole purpose is to facilitate the editing of attributes of components. 

\subsubsection*{sdb\_notes SDB's original comment in gattrib.c}
 In the spirit of open source/free software, major sections of gattrib's code were borrowed from other sources, and hacked together by SDB in Dec. 2003.\\ 
 Particularly rich sources for code were gEDA/gnetlist, and the gtkextra program testgtksheet.c.\\ 
\\ 
 Thanks to their authors for providing the foundation upon which this is built.\\ 
 Of course, I \textbf{*did*}
 write major portions of the code too . . . . .\\ 
 Some documentation about the internal operation of this program can be found in the ``NOTES'' file in the gattrib top-level directory.\\ 
 -- SDB December 2003 --\\ 
\\ 
\subsubsection*{Architecture}
 Extracted from SDB's mailing list posting:  \url{http://osdir.com/ml/cad.geda.devel/2007-06/msg00282.html} - believed to still be relevant.\\ 

  gattrib has three major components:\\ 
\begin{itemize}
\item It manipulates objects held in the TOPLEVEL data structure.\\ 
 It does this by importing structs and functions from libgeda. 
\item Gattrib defines its own layer of structs, notably SHEET\_DATA, which holds a table of attrib name=value pairs, and also holds a couple of linked lists corresponding to all component's refdeses, and to all attribute names found in the design.\\ 
 This stuff is native to gattrib. 
\item Gattrib uses a spreadsheet widget called GtkSheet.\\ 
 This stuff came from the GtkExtra project, which at one time offered a bunch of interesting widgets for graphing and visualization.\\ 
 I think they're still around; you can do a Google search for them.\\ 
 I stole the two .h files defining the spreadsheet widgets, and also stole code from the program itself to implement the run-time functions which deal with the spreadsheet. 

\end{itemize}

  When run, gattrib does this:\\ 
\begin{itemize}
\item It uses libgeda functions to read in your design, and fill up the TOPLEVEL struct. 
\item It then loops over everything in TOPLEVEL and fills out the refdes list and the attribute name list. It sticks these into a STRING\_LIST which is associated with the SHEET\_DATA struct. 
\item Then, knowing all the refdeses and all the attribute names, it creates a TABLE data struct (a member of SHEET\_DATA), and loops over each cell in the TABLE.\\ 
 For each cell, it queries TOPLEVEL for the corresponding name=value pair, and sticks the value in the TABLE. 
\item When done with that, it then creates a GtkSheet and populates it by looping over TABLE. 
\item Then it turns over control to the user, who can manipulate the GtkSheet.\\ 
 As the user adds and deletes values from the GtkSheet, the values are stored locally there.\\ 
 The GtkSheet is the master repository of all attributes at that point; the other data structures are not updated. 

\end{itemize}
 Saving out a design is similar, except the process runs in reverse:\\ 
\begin{itemize}
\item The program loops over all cells in GtkSheet, and sticks the values found into SHEET\_DATA.\\ 
 Handling issues like added/deleted columns happens first at the GtkSheet, and then to SHEET\_DATA and TOPLEVEL.\\ 
 I've kind of forgotten how I implemented these feaures, however. :-S 
\item Then, the program loops over the cells in SHEET\_DATA, and updates the attributes in TOPLEVEL using functions from libgeda, as well as by reaching directly into the TOPLEVEL data structure (a software engineering no-no).\\ 
 If a previously existing attrib has been removed, then it is removed from TOPLEVEL.\\ 
 If a new attrib has been attached to a component, then it is added to TOPLEVEL. 
\item Then the design is saved out using the save function from libgeda. 

\end{itemize}

  Therefore, think of SHEET\_DATA and the other gattrib data structures as a thin layer between GtkSheet and TOPLEVEL.\\ 
 The gattrib data structures are used basically for convenience while trying to build or update either of the two other, more important data structures.

